\begin{frame}{F 集合划分 by \itshape xiaolilsq}
	考虑 $1$ 以及大于 $\sqrt{n}$ 的质数都必须要单独放在一个集合内,而我们通过构造的方式证明其它数字最多额外剩下一个数,其余数都可以放在大小大于 $1$ 的集合中.
\end{frame}

\begin{frame}{F 集合划分 by \itshape xiaolilsq}
	具体而言,从大到小一次考虑每个小于等于 $\sqrt{n}$ 的质数 $p$,然后考虑 $p$ 所有还未被分配的倍数被分到哪些集合. \pause
	
	\begin{itemize}
		\item 如果 $p>2$ 且 $p(p+1)<n$,那么 $p$ 的所有还未被分配的倍数为 $p,2p,\dots,p^2$,此时我们可以将其全部分到同一个组内. \pause
		\item 如果 $p>2$ 且 $p(p+1)\ge n$,那么 $p$ 的倍数中至少有 $2p,3p,\dots,(p-1)p$ 和 $p(p+1)$,共有 $p-1$ 个还未被分配的质数,且它们的最小质因子都小于 $p$,亦或者说它们还可以在后面的过程中被分配。考虑那些不能在后面的过程中分配的数,也就是它们的最小质因子为 $p$,我们总是能够取这 $p-1$ 个还未被分配的数中的若干个补成 $p$ 的倍数,这样它们总能够在这一次分配过程中分好,而剩下的可以丢到后面的过程去考虑. \pause
		\item 如果 $p=2$,将所有数两个两个分组,最多会额外剩下一个,由此得证.
	\end{itemize}
\end{frame}

\begin{frame}{F 集合划分 by \itshape xiaolilsq}
	如果额外剩下一个,考虑通过细微调整的方法使其不会剩下. \pause
	
	具体而言就是在 $p=3$ 的分配过程时,如果有一个大小为 $3$ 的集合里面被分到的全都是 $6$ 的倍数,那么我们就可以把这些数全部放到 $p=2$ 的分配过程中,这样就不会剩下这样一个多出来的了. \pause
	
	实现后发现在 $n\ge 30$ 左右时总是可以这样调整,所以这样都是最优的。而不能调整的很少,可以通过证明如此构造得到的也是最优情况.
	
\end{frame}